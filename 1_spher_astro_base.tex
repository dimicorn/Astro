\begin{enumerate}
    \item Звезда Капелла ($\alpha$ Aur, прямое восхождение $\alpha = 5^h 16^m 41^s$, склонение $\delta = 46^\circ 00'$) кульминирует строго в зените. Определите высоты обеих кульминаций для звезды Мерак ($\beta$ UMa, $\delta= 56^\circ17'$) в этом же месте наблюдения.

    Решение:
    Найдём широту наблюдений, зная что Капелла кульминирует в зените:
    \begin{equation*}
        z = |\varphi - \delta| = 0 \Rightarrow \varphi = \delta = 46^\circ00'.
    \end{equation*}
    Зная широту наблюдений, мы можем найти обе кульминации для звезды Мерак:
    \begin{align*}
        h_\uparrow = 90^\circ - |\varphi - \delta| = 90^\circ - |46^\circ00' - 56^\circ 17'| = 79^\circ 43',\\
        h_\downarrow = -90^\circ + |\varphi + \delta| = -90^\circ + |46^\circ00' + 56^\circ 17'| = 12^\circ 17'.
    \end{align*}
    Ответ: $h_\uparrow = 79^\circ 43'$, $h_\downarrow = 12^\circ 17'$.
    \item Верхняя кульминимация звезды Вега ($\alpha$ Lyr, склонение $\delta = 38^\circ47'$) происходит на высоте $83^\circ 18'$. Определите широту места наблюдения. Определите высоту нижней кульминации.
    
    \item Нижняя кульминация звезды совпадает с горизонтом, а верхняя кульминация с зенитом. Определите широту места наблюдения и склонение звезды.
    \item Определите, какая из звезд поднимется выше над горизонтом для наблюдателя на широте $\varphi=55^\circ56'$. Мерак ($\beta$ UMa, $\delta= 56^\circ17'$) или Мицар ($\zeta$ UMa, $\delta = 54^\circ50'$).
    \item Определите склонение звезды, которая в Долгопрудном ($\varphi_1 = 55^\circ56'$) и во Владивостоке ($\varphi_2 = 43^\circ11'$) кульминирует на одно и той же высоте.
    \item Определите широты мест наблюдения, где звезда Фомальгаут ($\delta=-29^\circ37'$) является невосходящей. Рефракцией пренебречь.
    \item Определите широты места наблюдения, где звезды Капелла ($\alpha$ Aur, склонение $\delta=45^\circ59'53''$) и Бетельгейзе ($\alpha$ Ori, склонение $7^\circ24'25''$) одновременно являются невосходящими. А одновременно незаходящими?
    \item Оцените, существуют ли на небе такие звезды, что для заданной широты северного полушария их верхняя кульминация в два раза выше нижней? Найдите возможные диапазоны значений.
    \item В некоторый момент звезда со склонением $\delta = 30^\circ$ находилась в кульминации для наблюдателя в Санкт-Петербурге ($\varphi = 60^\circ$). В тот же момент вторая звезда оказалась также в кульминации, причем сумма высот звезд составила $125^\circ$. Определите склонение второй звезды.
    \item Две звезды на широте $\varphi = 23.5^\circ$ в верхней кульминации располагаются симметрично относительно зенита. Обе звезды заходящие. На какой минимальной высоте может происходить нижняя кульминация этих звёзд (до какой минимальной высоты может опуститься та из звёзд, которая опускается ниже)? Решение сопроводите чертежом.
    \item Склонение звезды А больше склонения звезды В в 2 раза. На какой широте верхняя кульминация этих звезд будет происходить на одном альмукантарате, если нижняя кульминация звезды А происходит на горизонте. Рефракцией пренебречь. Наблюдение проводятся в северном полушарии вдали от полюса.
    \item Северное полярное расстояние звезды А равно склонению звезды В. Верхняя кульминация звезды В происходит на той же высоте, что нижняя кульминация звезды А. Будет ли видно звезду В во время ее нижней кульминации, если наблюдатель находится в средней полосе России.
    \item У одной звезды зенитные расстояния в моменты верхней и нижней кульминации равны $20^\circ $ и $30^\circ$. А у второй звезды, наблюдаемой в том же месте, высота верхней кульминации $h= 80^\circ$. Определите высоту нижней кульминации второй звезды.
\end{enumerate}