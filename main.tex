\documentclass[a4paper, 12pt]{article}
\usepackage{placeins}
\usepackage[left=2cm,right=2cm,top=2cm,bottom=3cm,bindingoffset=0cm]{geometry}
\usepackage{indentfirst}
\usepackage[T2A]{fontenc}
\usepackage[utf8]{inputenc}
\usepackage[english,russian]{babel}
\usepackage[nottoc]{tocbibind}

\usepackage{wrapfig}
\usepackage{graphicx}
\graphicspath{{pictures/}}
\DeclareGraphicsExtensions{.pdf,.png,.jpg}

\usepackage{multirow}
\usepackage{pgfplots}
\pgfplotsset{compat=1.9}

\usepackage{amsfonts, amssymb, amsthm, mathtools, esvect}
\usepackage{enumitem}
\usepackage{amsmath}
\usepackage{physics}
\usepackage[unicode, pdftex]{hyperref}
\hypersetup{colorlinks, citecolor=blue, filecolor=blue, linkcolor=blue, urlcolor=blue}

\usepackage[version=4]{mhchem}
\usepackage{comment}
\title{\textbf{Решения задач по олимпиадной астрономии}}
\begin{document}
\author{Загоруля Д.С.}
\maketitle
\tableofcontents
\newpage
\section{Сферическая астрономия}
\subsection{Базовые задачи}
\begin{enumerate}
    \item Звезда Капелла ($\alpha$ Aur, прямое восхождение $\alpha = 5^h 16^m 41^s$, склонение $\delta = 46^\circ 00'$) кульминирует строго в зените. Определите высоты обеих кульминаций для звезды Мерак ($\beta$ UMa, $\delta= 56^\circ17'$) в этом же месте наблюдения.

    Решение:
    Найдём широту наблюдений, зная что Капелла кульминирует в зените:
    \begin{equation*}
        z = |\varphi - \delta| = 0 \Rightarrow \varphi = \delta = 46^\circ00'.
    \end{equation*}
    Зная широту наблюдений, мы можем найти обе кульминации для звезды Мерак:
    \begin{align*}
        h_\uparrow = 90^\circ - |\varphi - \delta| = 90^\circ - |46^\circ00' - 56^\circ 17'| = 79^\circ 43',\\
        h_\downarrow = -90^\circ + |\varphi + \delta| = -90^\circ + |46^\circ00' + 56^\circ 17'| = 12^\circ 17'.
    \end{align*}
    Ответ: $h_\uparrow = 79^\circ 43'$, $h_\downarrow = 12^\circ 17'$.
    \item Верхняя кульминимация звезды Вега ($\alpha$ Lyr, склонение $\delta = 38^\circ47'$) происходит на высоте $83^\circ 18'$. Определите широту места наблюдения. Определите высоту нижней кульминации.
    
    \item Нижняя кульминация звезды совпадает с горизонтом, а верхняя кульминация с зенитом. Определите широту места наблюдения и склонение звезды.
    \item Определите, какая из звезд поднимется выше над горизонтом для наблюдателя на широте $\varphi=55^\circ56'$. Мерак ($\beta$ UMa, $\delta= 56^\circ17'$) или Мицар ($\zeta$ UMa, $\delta = 54^\circ50'$).
    \item Определите склонение звезды, которая в Долгопрудном ($\varphi_1 = 55^\circ56'$) и во Владивостоке ($\varphi_2 = 43^\circ11'$) кульминирует на одно и той же высоте.
    \item Определите широты мест наблюдения, где звезда Фомальгаут ($\delta=-29^\circ37'$) является невосходящей. Рефракцией пренебречь.
    \item Определите широты места наблюдения, где звезды Капелла ($\alpha$ Aur, склонение $\delta=45^\circ59'53''$) и Бетельгейзе ($\alpha$ Ori, склонение $7^\circ24'25''$) одновременно являются невосходящими. А одновременно незаходящими?
    \item Оцените, существуют ли на небе такие звезды, что для заданной широты северного полушария их верхняя кульминация в два раза выше нижней? Найдите возможные диапазоны значений.
    \item В некоторый момент звезда со склонением $\delta = 30^\circ$ находилась в кульминации для наблюдателя в Санкт-Петербурге ($\varphi = 60^\circ$). В тот же момент вторая звезда оказалась также в кульминации, причем сумма высот звезд составила $125^\circ$. Определите склонение второй звезды.
    \item Две звезды на широте $\varphi = 23.5^\circ$ в верхней кульминации располагаются симметрично относительно зенита. Обе звезды заходящие. На какой минимальной высоте может происходить нижняя кульминация этих звёзд (до какой минимальной высоты может опуститься та из звёзд, которая опускается ниже)? Решение сопроводите чертежом.
    \item Склонение звезды А больше склонения звезды В в 2 раза. На какой широте верхняя кульминация этих звезд будет происходить на одном альмукантарате, если нижняя кульминация звезды А происходит на горизонте. Рефракцией пренебречь. Наблюдение проводятся в северном полушарии вдали от полюса.
    \item Северное полярное расстояние звезды А равно склонению звезды В. Верхняя кульминация звезды В происходит на той же высоте, что нижняя кульминация звезды А. Будет ли видно звезду В во время ее нижней кульминации, если наблюдатель находится в средней полосе России.
    \item У одной звезды зенитные расстояния в моменты верхней и нижней кульминации равны $20^\circ $ и $30^\circ$. А у второй звезды, наблюдаемой в том же месте, высота верхней кульминации $h= 80^\circ$. Определите высоту нижней кульминации второй звезды.
\end{enumerate}
\subsection{Олимпиадные задачи}
\begin{enumerate}
    \item Определите высоту горы на широте $68^\circ$, где не бывает полярной ночи.
    \item Древняя цивилизация построила на Земле (включая океаны) сеть сигнальных башен высотой 30 метров. С верхней площадки каждой башни были видны верхние площадки по крайней мере двух соседних башен. Зажигая на ней огни определенного цвета, можно было быстро передавать на большие расстояния весть об опасности. За какое минимальное время такую информацию можно было распространить по всей Земле, если время реакции солдата на башне, зажигающего огни, составляет 10 секунд? Атмосферным ослаблением света, рефракцией и рельефом Земли пренебречь.
    \item Любитель астрономии, не двигаясь по поверхности Земли, заметил, что заход Солнца за горизонт продолжался ровно 3 минуты. В каком географическом районе России он находился? Орбиту Земли считать круговой, атмосферной рефракцией пренебречь. Наблюдение происходит в сентябре месяца.
    \item Определите широту места наблюдения, где диск Солнца быстрее всего пересечет горизонт 1 января. Сколько времени на это будет происходить? Рефракцией пренебречь.
    \item Объект находится в точке осеннего равноденствия определите звездное время его захода за горизонт. Рефракцией пренебречь.
    \item Звезда Минтака ($\alpha = 5^h32^m$, $\delta=-0^\circ18'$) находится практически на небесном экваторе. Определите звездное время ее захода за горизонт. Определите его время захода за горизонт для наблюдателя из Долгопрудного $\varphi = 55^\circ56'$. Рефракцией пренебречь.
    \item В древнеримском войске ночь всегда делилась на 4 одинаковые стражи. Определите, во сколько раз отличалась продолжительность стражи в день зимнего солнцестояния от дня летнего солнцестояния. Рефракцией и размерами Солнца пренебречь. Широта Рима $\varphi= 42^\circ$.
    \item Азимут точки восхода Солнца в течении года меняется на $90^\circ$. Определите на какой широте это возможно? Рефракцией и угловыми размерами Солнца пренебречь.
    \item Определите длительность гражданских сумерек 23 апреля в Московской области ($\varphi=56^\circ$ с.ш.)
    \item Найдите азимут и высоту Солнца 22 июня в момент пересечения им первого вертикала для наблюдателя в Москве ($\varphi = 55^\circ45'$)?
    \item Звезда находится над горизонтом $11^h58^m$ и кульминирует на высоте $40^\circ$. Определите широту места наблюдения и склонение звезды.
    \item С какой части поверхности Земли можно наблюдать Международную космическую станцию, если известно, что высота ее круговой орбиты составляет 420 км, а наклонение $51.6^\circ$? Рефракцией, атмосферными помехами и сжатием Земли пренебречь.
    \item Составьте правильную последовательность моментов восходов Солнца 21 марта в этих точках.
    \begin{enumerate}[label=\alph*.]
        \item Килиманжаро $3^\circ4'$ южной широты, $37^\circ21'30''$ восточной долготы, высота 5881 м.
        \item Восточное побережье Африки $3^\circ4'$ южной широты, $40^\circ10'30''$ восточной долготы, высота 0 м.
        \item Восточное побережье Африки $2^\circ3'$ южной широты, $40\circ40'40''$ восточной долготы, высота 0 м.
        \item Восточное побережье Африки. г. Момбаса. Восточное побережье Африки $4^\circ2'$ южной широты, $39^\circ44'10''$ восточной долготы, высота 0 м.
    \end{enumerate}
    \item Какова продолжительность гражданских сумерек 7 июля в Петрозаводске ($\varphi = 61^\circ47'$)? Рефракцией и угловым размером Солнца пренебречь.
    \item Определите, в какие дни световой день, включая гражданские сумерки, равен ночи на широте Москвы.
    \item Определите, на какой широте можно одновременно наблюдать звезды Зубен Альгенуби $\alpha$ Lib $\alpha= 14^h51^m$  $\delta=-16^\circ06'$ (альфа Весов) и $\delta$ Козерога Денеб Альгеди $\alpha = 21^h47^m$ $\delta=-16^\circ08'$ на горизонте? Атмосферной рефракцией пренебречь.
    \item 21 декабря Луна, находящаяся в фазе первой четверти, заходит ровно в полночь. Определите время захода Луны на следующий день. Наблюдатель находится на широте Долгопрудного $\varphi = 56^\circ$.
\end{enumerate}

\section{Звёздное небо}
\begin{enumerate}
    \item Какое созвездие выпадает из следующего ряда: Южный Крест, Рысь, Индеец, Тукан, Павлин, Телескоп, Сетка?
    
    Ответ: \textbf{Рысь}, это единственное созвездие из северного полушария, остальные -- из южного.
    \item Какого <<рыбного>> созвездия не существует? Рыбы, Летучая рыба, Красная рыба, Золотая рыба, Южная рыба.

    Ответ: \textbf{Красная рыба}, такого созвездия не существует.
    \item Подберите каждой звезде из первой группы созвездие из второй группы:
    \begin{enumerate}[label=\alph*.]
        \item \textit{звезды}: Альдебаран, Бетельгейзе, Мицар, Вега, Денеб, Альтаир, Полярная, Регул;
        \item \textit{созвездия}: Орион, Большая Медведица, Малая Медведица, Телец, Лира, Орёл, Лебедь, Лев.
    \end{enumerate}

    Ответ:
    Альдебаран -- Телец, Бетельгейзе -- Орион, Мицар -- Большая Медведица, Вега -- Лира, Денеб -- Лебедь, Альтаир -- Орёл, Полярная -- Малая Медведица, Регул -- Лев.
    \item Как известно, созвездие Ориона в Подмосковье лучше всего наблюдать вечером в середине зимы. Как вы думаете, изменится ли ситуация для городов Канады, расположенных на той же широте? В какое время года Орион для канадцев вечером будет наблюдаться в таком же положении относительно горизонта, как и для жителей Подмосковья? Ответ объясните.

    Ответ: В формулах определяющих высоту кульминации нет зависимости на долготу, соответственно время кульминации будет отличаться, но высота кульминации будет такой же как у жителей Подмосковья. А разница во времени кульминаций будет компенсироваться разницей между локальными временами, поэтому \textbf{для городов Канады ситуация не изменится и Орион будет наблюдаться в том же положении в середине зимы}.
    \item Запишите, в каком созвездии можно наблюдать полную Луну 15 апреля.

    Ответ: во время полнолуния Луна находится в диаметрально противоположном направлении от Солнца. Солнце 15 апреля будет находится в Рыбах, а Луна соответственно \textbf{в Деве}.
    \item Запишите, в каком месяце Земля находится ближе всего к центру Галактики.

    Ответ: центр Галактики находится в направлении созвездия Стрельца, когда Солнце находится в созвездии Стрельца во время зимнего солнцестояния, то оно находится между центром Галактики и Землёй. Соответственно Земля находится ближе всего к центру Галактики через полгода, вблизи летного солнцестояния, то есть \textbf{в июне}.
\end{enumerate}

\end{document}
