\begin{enumerate}
    \item Определите высоту горы на широте $68^\circ$, где не бывает полярной ночи.
    \item Древняя цивилизация построила на Земле (включая океаны) сеть сигнальных башен высотой 30 метров. С верхней площадки каждой башни были видны верхние площадки по крайней мере двух соседних башен. Зажигая на ней огни определенного цвета, можно было быстро передавать на большие расстояния весть об опасности. За какое минимальное время такую информацию можно было распространить по всей Земле, если время реакции солдата на башне, зажигающего огни, составляет 10 секунд? Атмосферным ослаблением света, рефракцией и рельефом Земли пренебречь.
    \item Любитель астрономии, не двигаясь по поверхности Земли, заметил, что заход Солнца за горизонт продолжался ровно 3 минуты. В каком географическом районе России он находился? Орбиту Земли считать круговой, атмосферной рефракцией пренебречь. Наблюдение происходит в сентябре месяца.
    \item Определите широту места наблюдения, где диск Солнца быстрее всего пересечет горизонт 1 января. Сколько времени на это будет происходить? Рефракцией пренебречь.
    \item Объект находится в точке осеннего равноденствия определите звездное время его захода за горизонт. Рефракцией пренебречь.
    \item Звезда Минтака ($\alpha = 5^h32^m$, $\delta=-0^\circ18'$) находится практически на небесном экваторе. Определите звездное время ее захода за горизонт. Определите его время захода за горизонт для наблюдателя из Долгопрудного $\varphi = 55^\circ56'$. Рефракцией пренебречь.
    \item В древнеримском войске ночь всегда делилась на 4 одинаковые стражи. Определите, во сколько раз отличалась продолжительность стражи в день зимнего солнцестояния от дня летнего солнцестояния. Рефракцией и размерами Солнца пренебречь. Широта Рима $\varphi= 42^\circ$.
    \item Азимут точки восхода Солнца в течении года меняется на $90^\circ$. Определите на какой широте это возможно? Рефракцией и угловыми размерами Солнца пренебречь.
    \item Определите длительность гражданских сумерек 23 апреля в Московской области ($\varphi=56^\circ$ с.ш.)
    \item Найдите азимут и высоту Солнца 22 июня в момент пересечения им первого вертикала для наблюдателя в Москве ($\varphi = 55^\circ45'$)?
    \item Звезда находится над горизонтом $11^h58^m$ и кульминирует на высоте $40^\circ$. Определите широту места наблюдения и склонение звезды.
    \item С какой части поверхности Земли можно наблюдать Международную космическую станцию, если известно, что высота ее круговой орбиты составляет 420 км, а наклонение $51.6^\circ$? Рефракцией, атмосферными помехами и сжатием Земли пренебречь.
    \item Составьте правильную последовательность моментов восходов Солнца 21 марта в этих точках.
    \begin{enumerate}[label=\alph*.]
        \item Килиманжаро $3^\circ4'$ южной широты, $37^\circ21'30''$ восточной долготы, высота 5881 м.
        \item Восточное побережье Африки $3^\circ4'$ южной широты, $40^\circ10'30''$ восточной долготы, высота 0 м.
        \item Восточное побережье Африки $2^\circ3'$ южной широты, $40\circ40'40''$ восточной долготы, высота 0 м.
        \item Восточное побережье Африки. г. Момбаса. Восточное побережье Африки $4^\circ2'$ южной широты, $39^\circ44'10''$ восточной долготы, высота 0 м.
    \end{enumerate}
    \item Какова продолжительность гражданских сумерек 7 июля в Петрозаводске ($\varphi = 61^\circ47'$)? Рефракцией и угловым размером Солнца пренебречь.
    \item Определите, в какие дни световой день, включая гражданские сумерки, равен ночи на широте Москвы.
    \item Определите, на какой широте можно одновременно наблюдать звезды Зубен Альгенуби $\alpha$ Lib $\alpha= 14^h51^m$  $\delta=-16^\circ06'$ (альфа Весов) и $\delta$ Козерога Денеб Альгеди $\alpha = 21^h47^m$ $\delta=-16^\circ08'$ на горизонте? Атмосферной рефракцией пренебречь.
    \item 21 декабря Луна, находящаяся в фазе первой четверти, заходит ровно в полночь. Определите время захода Луны на следующий день. Наблюдатель находится на широте Долгопрудного $\varphi = 56^\circ$.
\end{enumerate}