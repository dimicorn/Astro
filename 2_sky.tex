\begin{enumerate}
    \item Какое созвездие выпадает из следующего ряда: Южный Крест, Рысь, Индеец, Тукан, Павлин, Телескоп, Сетка?
    
    Ответ: \textbf{Рысь}, это единственное созвездие из северного полушария, остальные -- из южного.
    \item Какого <<рыбного>> созвездия не существует? Рыбы, Летучая рыба, Красная рыба, Золотая рыба, Южная рыба.

    Ответ: \textbf{Красная рыба}, такого созвездия не существует.
    \item Подберите каждой звезде из первой группы созвездие из второй группы:
    \begin{enumerate}[label=\alph*.]
        \item \textit{звезды}: Альдебаран, Бетельгейзе, Мицар, Вега, Денеб, Альтаир, Полярная, Регул;
        \item \textit{созвездия}: Орион, Большая Медведица, Малая Медведица, Телец, Лира, Орёл, Лебедь, Лев.
    \end{enumerate}

    Ответ:
    Альдебаран -- Телец, Бетельгейзе -- Орион, Мицар -- Большая Медведица, Вега -- Лира, Денеб -- Лебедь, Альтаир -- Орёл, Полярная -- Малая Медведица, Регул -- Лев.
    \item Как известно, созвездие Ориона в Подмосковье лучше всего наблюдать вечером в середине зимы. Как вы думаете, изменится ли ситуация для городов Канады, расположенных на той же широте? В какое время года Орион для канадцев вечером будет наблюдаться в таком же положении относительно горизонта, как и для жителей Подмосковья? Ответ объясните.

    Ответ: В формулах определяющих высоту кульминации нет зависимости на долготу, соответственно время кульминации будет отличаться, но высота кульминации будет такой же как у жителей Подмосковья. А разница во времени кульминаций будет компенсироваться разницей между локальными временами, поэтому \textbf{для городов Канады ситуация не изменится и Орион будет наблюдаться в том же положении в середине зимы}.
    \item Запишите, в каком созвездии можно наблюдать полную Луну 15 апреля.

    Ответ: во время полнолуния Луна находится в диаметрально противоположном направлении от Солнца. Солнце 15 апреля будет находится в Рыбах, а Луна соответственно \textbf{в Деве}.
    \item Запишите, в каком месяце Земля находится ближе всего к центру Галактики.

    Ответ: центр Галактики находится в направлении созвездия Стрельца, когда Солнце находится в созвездии Стрельца во время зимнего солнцестояния, то оно находится между центром Галактики и Землёй. Соответственно Земля находится ближе всего к центру Галактики через полгода, вблизи летного солнцестояния, то есть \textbf{в июне}.
\end{enumerate}